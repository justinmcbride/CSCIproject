\hypertarget{index_information}{}\section{Information}\label{index_information}
This program collects data from the hardware sensors, then converts it to a readable unit (Celcius, etc), and uploads it to a server.

We only need one python file to accomplish this, and that file is \hyperlink{sensing_8py}{sensing.\-py}. This single file depends on the pypcduino library however, and that code is also included.

The appropriate documentation is included within.\hypertarget{index_authors}{}\section{Authors}\label{index_authors}
Made by\-:
\begin{DoxyItemize}
\item Justin Mc\-Bride
\item Austin Cerny
\item Reed Anderson
\item Aaron Holt
\end{DoxyItemize}\hypertarget{index_hardwareinfo}{}\section{Hardware Information}\label{index_hardwareinfo}
This python script runs on several pc\-Duino v2 boards located in different places. Each unit has various sensors attached to it, and for whatever sensor they have, the appropriate data is uploaded.\hypertarget{index_serverinfo}{}\section{Server Information}\label{index_serverinfo}
The server is a Ubuntu instance on Amazon's E\-C2 infrastructure. The server runs Mongo\-D\-B and listens for input through a R\-E\-S\-T A\-P\-I generated by a Dream\-Factory instance. The server also utilizes Angular.\-J\-S and\-Node.\-J\-S to create a graphical frontend for the information. 