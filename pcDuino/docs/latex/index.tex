\hypertarget{index_information}{}\section{Information}\label{index_information}
This program collects data from the hardware sensors, then converts it to a readable unit (Celcius, etc), and uploads it to a server.

We only need one python file to accomplish this, and that file is \hyperlink{sensing_8py}{sensing.\-py}. This single file depends on the pypcduino library however, and that code is also included.

The appropriate documentation is included within.\hypertarget{index_authors}{}\section{Authors}\label{index_authors}
Made by\-:
\begin{DoxyItemize}
\item Justin Mc\-Bride
\item Austin Cerny
\item Reed Anderson
\item Aaron Holt
\end{DoxyItemize}\hypertarget{index_hardwareinfo}{}\section{Hardware Information}\label{index_hardwareinfo}
This python script runs on several pc\-Duino v2 boards located in different places. Each unit has various sensors attached to it, and for whatever sensor they have, the appropriate data is uploaded.\hypertarget{index_serverinfo}{}\section{Server Information}\label{index_serverinfo}
The server is a Ubuntu instance on Amazon's E\-C2 infrastructure. The server runs Mongo\-D\-B and listens for input through a R\-E\-S\-T A\-P\-I generated by a Dream\-Factory instance. The server also utilizes Angular.\-J\-S to create a graphical frontend for the information.\hypertarget{index_requirements}{}\section{Requirements to Run}\label{index_requirements}
This code requires the external library 'requests'.\hypertarget{index_testing}{}\section{Testing}\label{index_testing}
Unit testing is enabled through the python unittest module. To run the tests, simply use the terminal command \$ python \hyperlink{test__sensing_8py}{test\-\_\-sensing.\-py}\hypertarget{index_static}{}\section{Static Analysis}\label{index_static}
This code utilizes the pylint module to perform static analysis on the code.

Simply run

\$ pylint \hyperlink{sensing_8py}{sensing.\-py}

to generate a report. A generated pylintrc directive file is included with the code to supress the following error messages\-:

{\itshape missing-\/docstring} -\/ Doxygen uses a different format than docstrings for documentation.

{\itshape invalid-\/name} -\/ We're not following Py\-Lint's Style guide

{\itshape mixed-\/indentation} -\/ Using spaces in Doxygen documentation confuses Py\-Lint

{\itshape anomalous-\/backslash-\/in-\/string} -\/ Doxygen uses backslashes in the documentation markdown.\hypertarget{index_documents}{}\section{Documentation}\label{index_documents}
The documentation for this code was created with the use of Doxygen. Because Doxygen does not natively (completely) support Python, the use of Doxypy from \href{http://code.foosel.org/doxypy}{\tt http\-://code.\-foosel.\-org/doxypy} is used as an input filter. 